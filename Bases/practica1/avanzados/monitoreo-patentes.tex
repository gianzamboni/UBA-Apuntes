%La Policía de la Ciudad de Buenos Aires desea implementar un sistema de sensores para patentes con el objetivo de detectar vehículos robados o con orden de captura. Los sensores se instalarán en diferentes puntos de la ciudad.\\
%De cada sensor se precisa saber el identificador, la calle y altura donde está instalado, el barrio y el código postal.\\
%Las calles y barrios deben estar tabulados para luego poder listar la ruta que ha seguido el vehículo detectado. \\
%Los sensores están conectados en forma inalámbrica a un destacamento policial. Los destacamentos poseen un nombre, dirección, teléfono y la comuna de la ciudad a la cual pertenecen. Se debe contar con el listado de vehículos que se deseen
%detectar, ya sea vehículos robados o con orden de detención. De todos ellos se guarda la patente, marca, modelo, color, y datos del dueño (DNI, nombre y apellido, domicilio y un teléfono de contacto). Para los vehículos robados
%además se precisa saber la fecha de la denuncia por robo. \\
%Cuando un vehículo que cumple con estas condiciones pasa por el sensor, se dispara una alarma en el destacamento asociado al sensor con la fecha y hora de la detección.\\
%Se desea saber:
%\begin{itemize}
%	
%	\item Todas las alarmas disparadas en un rango de fechas
%	\item El ranking de los 10 destacamentos policiales con más alarmas detectadas
%	\item Los datos del vehículo que causa una alarma en particular
%	\item La ruta que siguió un vehículo desde que un sensor lo detectó (en caso que haya pasado en su trayecto por otros sensores
%	
%\end{itemize}
%Realice el \textbf{DER}, con sus restricciones en lenguaje natural, que modele el problema planteado. Realizar el \textbf{Modelo Relacional}. Se debe consignar todas las relaciones que surjan del DER, todos los atributos, todas las claves primarias y todas las claves foráneas.

\begin{center}
	\scalebox{0.8}{
		\begin{tikzpicture}[DER]	
			\node[entity] (sensor)	{Sensor};
			\node[relationship] (conectadoA) [right=of sensor] {conectado a};
			\node[entity]		(destacamento) [right=of conectadoA] {Destacamento};
			\node[relationship] (ubicadoEn)	[below=of sensor] {ubicado en};
			\node[entity]		(ubicacion)  [below=of ubicadoEn] {Ubicación};
			\node[relationship] (detectadoPor) [left=of sensor] {detectadoPor};
			\node[entity]		(vehiculo)	[left=of detectadoPor]	{Vehiculo};
			\node[relationship]	(perteneceA) [below=of vehiculo] {pertenece a};
			\node[entity]		(persona)	[below=of perteneceA] {Persona};
			\node[inheritance]	(inhVehiculo) [left=of vehiculo] {D};
			\node[entity]		(robado)		[above left=of inhVehiculo] {Robado};
			\node[entity]		(conOrdenDeDetencion)		[below left=of inhVehiculo] {Con Orden de Detencion};
			
			%% Atributos destacamento
			\node[attributePK] (destacamentoIdDestacamento) [above=of destacamento] {\PK{idDestacamento}};
			\node[attribute]	(destacamentoNombre) [above right=of destacamento] {nombre};
			\node[attribute] (destacamentoDireccion) [right=of destacamento] {dirección};
			\node[attribute] (destacamentoTelefono) [below right=of destacamento] {teléfono};
			\node[attribute] (destacamentoComuna)	[below=of destacamento] {comuna};
			
			%% Atributos persona
			\node[attributePK] (personaDni)	[above right=of persona] {\PK{dni}};
			\node[attribute] (personaNombre)	[right=of persona] {nombre};
			\node[attribute] (personaApellido)	[below right=of persona] {apellido};
			\node[attribute] (personaDomicilio)	[below=of persona] {domicilio};
			\node[attribute] (personaTelefono)	[below left=of persona] {telefono};
			
			%% Atributos robado
			\node[attribute] (robadoFechaDenuncia) [left=of robado]	{fechaDenuncia};
			
			%% Atributos sensor
			\node[attributePK] (sensorIdSensor)	[above=of sensor]	{\PK{idSensor}};
			
			%%  Atributos ubicacion
			\node[attributePK] (ubicacionCalle) [above right=of ubicacion] {\PK{calle}};
			\node[attributePK] (ubicacionAltura) [right=of ubicacion] {\PK{altura}};
			\node[attribute]	(ubicacionBarrio) [below right=of ubicacion] {barrio};
			\node[attribute]	(ubicacionCodPostal) [below=of ubicacion] {codPostal};
			
			% Atributo vehiculo
			\node[attributePK] (vehiculoPatente) [above left=of vehiculo, yshift=1cm] {\PK{patente}};
			\node[attribute] (vehiculoMarca) [above=of vehiculo] {marca};
			\node[attribute] (vehiculoModelo) [above right=of vehiculo, yshift=1.5cm] {modelo};
			\node[attribute] (vehiculoColor) [below right=of vehiculo] {color};

			%% Atributo detectadoPor
			\node[attributePK]	(detectadoPorFecha) [above=of detectadoPor] {fecha};
			\node[attributePK]	(detectadoPorHora) [above right=of detectadoPor] {hora};
			\path
				%% Atributos destacamento
				(destacamento) edge (destacamentoIdDestacamento)
				(destacamento) edge (destacamentoNombre)
				(destacamento) edge (destacamentoDireccion)
				(destacamento) edge (destacamentoTelefono)
				(destacamento) edge (destacamentoComuna)
				
				%% Atributos persona
				(persona) edge (personaDni)
				(persona) edge (personaNombre)
				(persona) edge (personaApellido)
				(persona) edge (personaDomicilio)
				(persona) edge (personaTelefono)
				
				%%Atributos robado
				(robado) edge (robadoFechaDenuncia)
				
				%% Atributos sensor
				(sensor) edge (sensorIdSensor)
				
				%% Atributos ubicacion
				(ubicacion) edge (ubicacionCalle)
				(ubicacion) edge (ubicacionAltura)
				(ubicacion) edge (ubicacionBarrio)
				(ubicacion) edge (ubicacionCodPostal)
			
				%% Atributos vehiculo
				(vehiculo) edge (vehiculoPatente)
				(vehiculo) edge (vehiculoMarca)
				(vehiculo) edge (vehiculoModelo)
				(vehiculo) edge (vehiculoColor)
				
				%% Inheritance vehiculo
				(inhVehiculo) edge node[above, pos=0.5]{tipo} (vehiculo)
				(inhVehiculo) edge (robado)
				(inhVehiculo) edge (conOrdenDeDetencion)
				
				%%Relacion conectadoA
				(conectadoA) edge node[above, near end]{N} (sensor)
				
				%% Relacion detectado por
				(detectadoPor) edge (detectadoPorFecha)				
				(detectadoPor) edge (detectadoPorHora)		
				%% Relacion pertenece a
				(perteneceA) edge node[right, near end]{N} (vehiculo)
				
				%% Relación ubicado en
				(ubicadoEn) edge node[right, near end]{1} (sensor)

			;
			
			\path[rightOptional]
				%% Relación conectado a
				(conectadoA) edge node[above, near end]{1} (destacamento)
				
				%% Relación detectado por
				(detectadoPor) edge node[above, near end]{N} (sensor)
				(detectadoPor) edge node[above, near end]{M} (vehiculo)
				
				%% Relacion pertenece a
				(perteneceA) edge node[right, near end]{1} (persona)
				
				
				%% Relación ubicado en
				(ubicadoEn) edge node[right, near end]{1} (ubicacion)
			;
		\end{tikzpicture}	
	}
\end{center}

\begin{multicols}{2}
	\paragraph{MER}
	\begin{itemize}
		\item[] Sensor(\PK{idSensor}, \FK{calle, altura}, \FK{idDestacamento})
		\item[] Destacamento(\PK{idDepartamento}, nombre, direccion, telefono, comuna)
		\item[] Ubicacion(\PK{calle}, \PK{altura}, barrio, codPostal)
		\item[] Vehiculo(\PK{patente}, marca, modelo, color, \FK{dni})
		\item[] Persona(\PK{dni}, nombre, apelildo, domicilio, telefono)
		\item[] Robado(\FPK{patente}, fechaDenuncia)
		\item[] detectadoPor(\FPK{sensor}, \FPK{patente}, \PK{fecha}, \PK{hora})
	\end{itemize}
	
	\columnbreak
	
	\paragraph{Consultas}
	\begin{itemize}
		\item Seleccionar todas las alarmas en el rango de fechas deseado desde \textit{detectadoPor}.
		\item Join con \textit{dectadoPor}, \textit{Sensor} y \textit{Destacamento}. Group by destacamento y count de la cantidad de entradas de cada uno de ellos. Me quedo con los destacamentos con los diez destacamentos con count más alto.
		\item Join \textit{detectadoPor} y \textit{vehiculo}. Filtrar la entrada correspondiente a la alarma deseada.
		\item Filtaramos \textit{dectadoPor} por el vehiculo deseado. Ordenando las entradas por fecha y hora.
	\end{itemize}
\end{multicols}
