%Una empresa que fabrica perfumes, precisa organizar su base de productos. Cada producto está codificado, tiene un nombre y un costo que surge de un cálculo donde intervienen los costos de las materias primas y otros productos que lo componen. Un producto está compuesto por otros productos y/o materias primas. Las materias primas poseen un costo base. \\
%Un cliente, del que se desea guardar el CUIT, Razón social, Domicilio, Teléfono y tipo de IVA, realiza pedidos de productos. Los pedidos están numerados correlativamente, poseen una descripción, una fecha y tienen ítems con los productos. El cliente solicita una determinada cantidad de cada producto y en cada ítem del pedido se detalla además el precio unitario.\\
%No se pueden realizar pedidos de materias primas.\\
%Para cumplir con los pedidos de clientes, la empresa realiza órdenes de fabricación. Estas órdenes están codificadas por número de partida. También poseen una fecha de la orden y una de vencimiento de la partida. La fecha de vencimiento debe ser mayor a la fecha de la orden. En una orden de fabricación pueden fabricarse varios ítems de diversos pedidos. No se pueden fabricar materias primas. Sólo se realizan fabricaciones de los ítems pedidos.
%
%Realice el DER, con sus restricciones en lenguaje natural, que modele el problema planteado.
%El Modelo deber\'{a} permitir adem\'{a}s resolver las siguientes consultas:
%\begin{itemize}
%	
%	\item El costo final de cualquier producto
%	\item Para cada cliente, la lista de los productos pedidos en una fecha con la fecha de vencimiento de la partida
%	\item El costo de fabricar una orden para un cliente
%	\item La composición de un producto
%	\item Las ordenes de fabricación que cumplen con fabricar un pedido dado
%	
%\end{itemize}
%Realizar el Modelo Relacional. Se debe consignar todas las relaciones que surjan del DER, todos los atributos, todas las claves primarias y todas las claves foráneas
\vspace*{-2.5cm}
\begin{center}
	\scalebox{0.8}{
		\begin{tikzpicture}[DER]	
		\node[entity] (producto)										{Producto};
		\node[inheritance] (inhProducto) [below=of producto] 			{D};
		\node[entity] (materiaPrima)	[below left=of inhProducto] 	{Materia Prima};
		\node[entity] (compuesto)		[below right=of inhProducto, yshift=-2cm] 	{Compuesto};
		\node[relationship] (compuestoPor) [right=of compuesto, xshift=4cm]			{compuesto por};
		\node[relationship]	(pedidoPor)		[below=of compuesto]		{pedido por};
		\node[weak entity]		(pedido)		[below=of pedidoPor, yshift=-1cm]	{Pedido};
		\node[weak relationship] (realizadoPor) [right=of pedido, xshift=4cm]		{realizado por};
		\node[entity] (cliente)		[right=of realizadoPor]				{Cliente};
		
		\draw ($(compuesto.north west) + (-3cm, 0.5cm)$) rectangle ($(pedido.south east) + (3.75cm, -2.25cm)$);
		
		\node[relationship] (resuelve)	[left=of pedidoPor, xshift=-4cm]			{resuelve};
		\node[entity]		(ordenFabricacion) [left=of resuelve]		{Orden de Fabricacion};
		
		%% Atributos cliente
		\node[attributePK]		(clienteDni)		[above=of cliente] {\PK{dni}};
		\node[attribute]		(clienteRazonSocial)[above right=of cliente] {razonSocial};
		\node[attribute]		(clienteDomicilio)	[right =of cliente] {domicilio};
		\node[attribute]		(clienteTelefono)	[below right=of cliente] {telefono};
		\node[attribute]		(clienteTipoIVA)	[below=of cliente] {tipoIVA};

		%% Atributo ordenFabricacion
		\node[attributePK]		(ordenNroPartida)	[above=of ordenFabricacion] {nroPartida};
		\node[attribute]		(ordenFecha)		[left=of ordenFabricacion] {fecha};
		\node[attribute]		(ordenVencimiento)	[below=of ordenFabricacion] {vencimiento};
		
		%% Atributos pedido
		\node[attributePK]		(pedidoNroPedido)	[below left=of pedido]	{\PK{nroPedido}};
		\node[attribute]		(pedidoDescripcion)	[below =of pedido]	{descripcion};
		\node[attribute]		(pedidoFecha)		[below right=of pedido]	{fecha};

		%% Atributos producto
		\node[attributePK] (productoCodigo) [above left=of producto] {\PK{código}};
		\node[attribute]   (productoCosto)  [left=of producto] {costo};
		\node[attribute]   (productoNombre) [below left=of producto] {nombre};
		
		%% Atributos pedidoPor
		\node[attribute]	(pedidoPorCantidad) [right=of pedidoPor] {cantidad};
		\node[attribute]	(pedidoPorPrecioUnitario) [below right=of pedidoPor] {precioUnitario};
		
		\path
		
		%% Atributos cliente
		(cliente) edge (clienteDni)		
		(cliente) edge (clienteRazonSocial)
		(cliente) edge (clienteDomicilio)	
		(cliente) edge (clienteTelefono)	
		(cliente) edge (clienteTipoIVA)	
		
		%% Atributos orden de fabricación
		(ordenFabricacion) edge (ordenNroPartida)
		(ordenFabricacion) edge (ordenFecha)
		(ordenFabricacion) edge (ordenVencimiento)

		%% Atributos pedido
		(pedido) edge (pedidoNroPedido)
		(pedido) edge (pedidoDescripcion)
		(pedido) edge (pedidoFecha)
		
		%% Atributos producto
		(producto) edge (productoCodigo)
		(producto) edge (productoCosto)
		(producto) edge (productoNombre)
				
		%% Inheritance producto
		(inhProducto) edge node[right, pos=0.5]{tipo} 	(producto)
		(inhProducto) edge 								(materiaPrima)
		(inhProducto) edge 								(compuesto)
		
		%% Relación compuesto por
		(compuestoPor) edge node[above, near end]{N}	(compuesto)
		(compuestoPor) edge[out=east, in=north] node[left, pos=0.95]{M}		(producto)
		
		%% Relacion pedido por
		(pedidoPor) edge node[right, near end]{N}	(pedido)
		(pedidoPor) edge (pedidoPorPrecioUnitario)
		(pedidoPor) edge (pedidoPorCantidad)
		
		%% Relacion realizado por
		(realizadoPor) edge node[above, near end]{N} (pedido)
		
		%% Relacion resuelve
		(resuelve) edge node[above, near end]{N} (ordenFabricacion)
		;
		
		
		\path[rightOptional]
			%% Relacion pedido por
			(pedidoPor) edge node[right, near end]{N}	(compuesto)
			
			%% Relacion realizado por
			(realizadoPor) edge node[above, near end]{1} (cliente)
			
			%% Relacion resuelve
			(resuelve) edge node[above, near end]{M} ($(pedidoPor.west) + (-2.51cm, 0)$)
		;
		\end{tikzpicture}	
	}
\end{center}

\newpage
\begin{multicols}{2}
	\paragraph{MER}
	\begin{itemize}
		\item[] Producto(\PK{codigo}, costo, nombre, tipo)
		\item[] Compuesto(\FPK{codigo})
		\item[] Cliente(\PK{dni}, razonSocial, domicilio, telefono, tipoIVA)
		\item[] Pedido(\FPK{dni}, \PK{nroPedido}, descripcion, fecha)
		\item[] OrdenDeFabricacion(\PK{nroPartida}, fecha, vencimiento)
		\item[] compuestoPor(\FPK{codigoCompuesto}, \FPK{codigoProducto})
		\item[] pedidoPor(\FPK{codigo}, \FPK{dni, nroPedido}, cantidad, precioUnitario)
		\item[] resuelve(\FPK{nroPartida}, \FPK{codigo, dni, nroPedido})
	\end{itemize}
	\columnbreak
	\paragraph{Restricciones adicionales}
	\begin{itemize}
		\item Un compuesto no puede estar compuesto por si mismo.
		\item La fecha de una partida es menor a su fecha de vencimiento.
	\end{itemize}
	\paragraph{Consultas}
	\begin{itemize}
		\item El costo final de un producto es el atributo \textit{costo} de \textit{Producto}. Solamente hay que hacer un select.
		\item Join con \textit{cliente}, \textit{pedido}, \textit{pedidoPor}, \textit{resuelve} y \textit{ordenDeFabricacion}. Filtarmos por el cliente y la fecha deseada.
		\item Dado un pedido, hacemos join con \textit{Pedido}, \textit{pedidoPor}, \textit{Compuesto}, \textit{Producto}. Acá tenemos todos los datos para sumar los costos de fabricar las cantidades necesarias de cada producto ($Producto.costo\times pedidoPor.cantidad$).
		\item Join de \textit{pedidoPor} con \textit{resuelve} y seleccionamos todas las entradas que corresponden al pedido deseado.
	\end{itemize}
\end{multicols}
