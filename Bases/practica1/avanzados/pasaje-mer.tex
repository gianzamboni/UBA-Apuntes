%%Considere los siguientes diagramas de entidad-relacion y para cada uno de ellos ima-gine un problema donde tenga sentido esa forma de modelado. Explique su elección. Realice el modelo relacional de cada uno y compárelos. Justifique las respuestas.

%\begin{figure}[h]
%	\centering
%	\includegraphics[scale=0.55]{img/Diagrama1}
%\end{figure}
%
%\begin{figure}[h]
%	\centering
%	\includegraphics[scale=0.60]{img/Diagrama2}
%\end{figure}
%
%\begin{figure}[h]
%	\centering
%	\includegraphics[scale=0.60]{img/Diagrama3}
%\end{figure}


\begin{multicols}{2}
	
	\noindent\textbf{MER 1}
	
	\noindent Persona(\PK{dni}, nombre, direccion, telefono, obraSocial, nroAfiliado, nombreSindicato, fechaIngreso, puesto)
	
	\noindent tieneHijos(\FPK{dni1}, \FPK{dni2})
	
	\vspace*{0.5cm}
	\noindent\textbf{Restricción:} Una persona no puede ser su propio hijo ($dni1 \neq dni2$)
	\vspace*{1cm}
	
	\noindent\textbf{MER2}
	
	\noindent Persona(\PK{dniPersona}, nombre, direccion, telefono, obraSocial, nroAfiliado, nombreSindicato, fechaIngreso, puesto)
	
	\noindent Hijos(\PK{dniHijo}, nombre, direccion)
	
	\noindent tiene(\FPK{dniPersona}, \FPK{dniHijo})
	
	\columnbreak
	\noindent\textbf{MER3}
	
	\noindent Persona(\PK{dni}, nombre, direccion, tipo)
	
	\noindent Empleado(\FPK{dni}, nombreSindicato, nroAfiliado, obraSocial, telefono, fechaIngreso, puesto)
	
	\noindent Hijos(\FPK{dni})
	
	\noindent tiene(\FPK{dni1}, \FPK{dni2})
	
	\vspace*{0.5cm}
	\noindent\textbf{Restriccion:} \textit{dn1} debe ser el dni de un empleado y \textit{dn2} el dni de un hijo. (dn1 debe tener una entrada en la tabla empleado y dn2 una en la tabla hijo)
\end{multicols}

