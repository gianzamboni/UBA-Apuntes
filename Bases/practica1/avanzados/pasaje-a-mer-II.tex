%Dado el siguiente diagrama de entidad-relacion imagine una situación para que este diagrama tenga sentido y descríbala a modo de requerimiento. Realice el Modelo Relacional asociado al diagrama
%
%\pagebreak
%
%\begin{figure}[h]
%	\centering
%	\includegraphics[scale=0.60]{img/MEREmpresa}
%\end{figure}

\textbf{MER}

Producto(\PK{idProducto}, descripcion)

Distribuidor(\PK{numDistribuidor}, nombre)

Area(\PK{idArea}, nombre)

Local(\PK{numLocal}, direccion)

Deposito(\PK{numDeposito}, \FPK{Local} capacidad)

Empleado(\PK{CUIL}, fechaIngreso, tipoContratacion)

EmpleadoEfectivo( \FPK{CUIL}, cantHijos)

EmpleadoEfectConGremio(\FPK{CUIL}, fechaAfiliacionGremio)

EmpleadoEfectConPrepaga(\FPK{CUIL}, numAfiliado)

trabajaEn(\FPK{numLocal}, \FPK{CUIL}, \FK{idArea}, cantHoras)

distribuidoPor(\FPK{idProducto}, \FPK{numDistribuidor}, \FK{idArea})
