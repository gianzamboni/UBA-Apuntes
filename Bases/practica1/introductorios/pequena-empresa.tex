%Una pequeña empresa precisa guardar información acerca de clientes, artículos y
%pedidos. Hasta el momento se registran los siguientes datos en documentos varios:
%\begin{enumerate}
%	\item Para cada cliente: Número de cliente (único), Direcciones de envío (varias por cliente), Saldo, Límite de crédito (depende del cliente, pero en ningún caso debe superar los \$3.000.000 ), Descuento.
%	\item Para cada artículo: Número de artículo (único), Fábricas que lo distribuyen, Exis-tencias de ese artículo en cada fábrica, Descripción del artículo.
%	\item Para cada pedido: Una cabecera y el cuerpo del pedido. La cabecera está formada por el número de cliente, dirección de envío y fecha del pedido. El cuerpo del pedido está compuesto de varias líneas y en cada línea se especifican el número del artículo pedido y la cantidad.
%\end{enumerate}
%Además, se ha determinado que se debe almacenar la información de las fábricas. Sin embargo, dado el uso de distribuidores, se usará: Número de la fábrica (único) y Teléfono de contacto. Y se desean ver cuántos artículos (en total) provee la fábrica. También, por información estratégica, se podría incluir información de fábricas alternativas respecto de las que ya fabrican artículos para esta empresa.\\
%\textbf{Nota}: Una dirección se entenderá como Nº, Calle, Comuna y Ciudad. Una fecha incluye hora.

\begin{center}
	\textbf{DER}
	\vspace*{1cm}
	
	\scalebox{1}{
		\begin{tikzpicture}[.style={DER}]	
		%% Entidades y relaciones
		\node [agregacion] (agregacionClienteDireccion) {
			\begin{tikzpicture}
			%% Entidades y relaciones
			\node[entity]		(cliente)											{Cliente};
			\node[relationship]	(enviaA)				[right=of cliente]			{envia a};
			\node[entity]		(direccion)				[right=of enviaA]			{Direccion};
			
			%%Atributos Cliente
			\node[attributePK]	(clienteNroCliente)		[above=of cliente] 			{\PK{nroCliente}};
			\node[attribute] 	(clienteSaldo)			[above left=of cliente]		{saldo};
			\node[attribute] 	(clienteLimiteCredito)	[left=of cliente]			{limiteCredito};
			\node[attribute] 	(clienteDescuento)		[below left=of cliente]		{descuento};
			
			%% Atributos direccion
			\node[attributePK]	(direccionIdDireccion)	[above=of direccion] 		{\PK{idDireccion}};
			\node[attribute] 	(direccionN)			[above right=of direccion]	{numero};
			\node[attribute] 	(direccionCalle)		[right=of direccion]			{calle};
			\node[attribute] 	(direccionComuna)		[below right=of direccion]	{comuna};
			\node[attribute] 	(direccionCiudad)		[below=of direccion]		{ciudad};
			
			\path
			%%Atributos cliente
			(cliente)	edge	(clienteNroCliente)
			(cliente) 	edge	(clienteSaldo)
			(cliente) 	edge	(clienteLimiteCredito)
			(cliente) 	edge	(clienteDescuento)
			
			%% Atributos direccion
			(direccion)	edge	(direccionIdDireccion)
			(direccion)	edge	(direccionN)	
			(direccion)	edge	(direccionCalle)	
			(direccion)	edge	(direccionComuna)	
			(direccion)	edge	(direccionCiudad)	
			
			%%Relacion envia A
			(enviaA) 	edge	node[yshift=0.2cm, pos=0.8] {N} 	(cliente)
			(enviaA) 	edge	node[yshift=0.2cm, near end] {M} 	(direccion)
			
			;
			\end{tikzpicture}
		};
		
		\node[relationship]	(esCabeceraDe)	[below=of agregacionClienteDireccion] 	{cabecera de};
		\node[entity]		(pedido)		[below=of esCabeceraDe]					{Pedido};
		\node[] 			(auxI)			[right=of pedido] 						{};  %%Auxiliar
		\node[relationship] (pide)			[right=of auxI]							{pide};
		\node[] 			(auxII)			[right=of pide] 						{};  %%Auxiliar
		\node[]				(auxIII)		[right=of auxII]						{};	 %%Auxiliar
		\node[entity]		(articulo)		[right=of auxIII]						{Articulo};
		\node[relationship]	(distribuye)	[above=of articulo]						{distribuye};
		\node[entity]		(fabrica)		[above=of distribuye]					{Fabrica};
		
		
		%%Atributos Articulo
		\node[attributePK]	(articuloNroArticulo)	[right=of articulo] 					{\PK{nroArticulo}};
		\node[attribute]	(articuloDescripcion)	[above right=of articulo] 				{descripcion};
		
		%%Atributos Fabrica
		\node[attributePK]	(fabricaNroFabrica)		[above=of fabrica] 						{\PK{nroFabrica}};
		\node[attribute]	(fabricaTelefono)		[above right=of fabrica] 				{telefono};
		
		%%Atributos de pedido
		\node[attributePK]	(pedidoIdPedido)		[left=of pedido] 						{\PK{idPedido}};
		\node[attribute]	(pedidoFecha)			[above left=of pedido] 					{fecha};
		\node[attribute] 	(pedidoFechaDia)	[above left=of pedidoFecha, yshift=-1cm] {dia};
		\node[attribute] 	(pedidoFechaHora)	[below left=of pedidoFecha, yshift=1cm] {hora};
		
		%% Atributos distribuye
		\node[attribute]	(distribuyeCantDisponible)	[left=of distribuye] 	{cantDisponible};
		
		%%Atributos pide
		\node[attribute]	(pideCantidad)				[below=of pide] 			{cantidad};
		
		%% Atributos PedidoFecha
		\path
		%% Atributos articulo
		(articulo)		edge	(articuloNroArticulo)
		(articulo)		edge	(articuloDescripcion)
		
		%% Atributos pedido
		(pedido)		edge 	(pedidoIdPedido)
		(pedido)		edge	(pedidoFecha)
		(pedidoFecha)	edge	(pedidoFechaDia)
		(pedidoFecha)	edge	(pedidoFechaHora)
		
		%% Atributos fabrica
		(fabrica)		edge	(fabricaNroFabrica)
		(fabrica)		edge	(fabricaTelefono)
		
		%% Atributos pide
		(pide) 			edge 	(pideCantidad)
		
		%%Atributos distribuye
		(distribuye)	edge	(distribuyeCantDisponible)
		
		%% Relacion distribuye
		(distribuye) 		edge	node[right, near end] {M}	(articulo)
		
		%% Relacion es cabecera de
		(esCabeceraDe)		edge	node[right, near end] {M}	(pedido)
		
		%% Relacion pide
		(pide)				edge	node[above, near end] {N} 	(pedido)
		(pide)				edge 	node[above, near end] {M} 	(articulo)
		;
		
		\path[rightOptional]
		(esCabeceraDe) 	edge 	node[right, near end] {1} 	(agregacionClienteDireccion)
		(distribuye) 	edge	node[right, near end] {N}	(fabrica)
		;
	\end{tikzpicture}
	
}
\end{center}

\textbf{Restricciones adicionales:}
\begin{itemize}
\item \textit{Cliente.limiteCredito} debe ser menor a 3.000.000
\end{itemize}

\begin{multicols}{2}
\textbf{MER}

Cliente(\PK{nroCliente}, saldo, limiteCredito, descuento)

Direccion(\PK{idDireccion}, numero, calle, comuna, ciudad)

enviaA(\FPK{nroCliente}, \FPK{idDireccion})

Pedido(\PK{idPedido}, dia, hora, \FK{nroCliente, idDireccion})

Articulo(\PK{nroArticulo}, descripcion)

pide(\FPK{idPedido}, \FPK{nroArticulo}, cantidad)

Fabrica(\PK{nroFabrica}, telefono)

distribuye(\FPK{nroFabrica}, \FPK{nroArticulo}, cantDisponible)

\columnbreak
\red{\textbf{Notas}}
\begin{itemize}
	\item Al realizar una agregación de una relación M-N, se la toma como a una entidad más. Puede participar en cualquier tipo de relación y cuantas relaciones sea necesario.
	\item Se pide saber cuantos artículos en total provee la fábrica. No es necesario almacenar este dato, se puede hacer una búsqueda en la tabla \textit{distribuye} de todos los objetos que tienen el \textit{nroFabrica} adecuado y sumar todos los valores.
	\item Fábrica tiene participación parcial en la relación distribuye. Esto quiere decir que habrá algunas qué no distribuirán ningún producto a la empresa (estas son las fábricas alternativas).
\end{itemize}

\end{multicols}
