%\begin{enumerate}
%	\item Una empresa contrata programadores y al momento del ingreso, éstos deben explicitar qué lenguajes de programación conocen. En ese mismo momento la empresa le toma un examen a cada programador para cada lenguaje de programación que conozca. Es necesario, además, llevar registro de la calificación obtenida. Para evaluar a los entrevistados, la empresa cuenta con exámenes estándar preparados, de los que se conoce la cantidad de ejercicios, el nivel de dificultad, la fecha de creación y el texto del examen.
%	\item	¿Cómo modificaría el modelo conceptual del punto (a) si el examen se le tomara sólo a algunos de los programadores (tener en cuenta que sigue siendo de interés registrar los lenguajes conocidos por cada entrevistado)?
%\end{enumerate}

\subsubsection*{a)}
\begin{center}
	\textbf{DER}
	\vspace*{1cm}
	
	\scalebox{1}{
		\begin{tikzpicture}[.style={DER}]	
		%% Entidades y realaciones
		\node[entity]		(programador)												{Programador};
		\node[relationship] (conoce)					[right=of programador]			{conoce};
		\node[entity]		(lenguaje)					[right=of conoce]				{Lenguaje};
		\node[entity]		(examen)					[above=of conoce]				{Examen};
		
		%%Atributos lenguaje
		\node[attributePK]	(lenguajeIdLenguaje)		[right=of lenguaje]			{\PK{idLenguaje}};
		\node[attribute]	(lenguajeNombre)			[below right=of lenguaje]		{nombre};
		
		%%Atributos examen
		\node[attribute]	(examenDificultad)			[above= of examen]				{dificultad};
		\node[attribute]	(examenCantEjercicios)		[left=of examenDificultad]		{cantEjercicios};
		\node[attributePK]	(examenIdExamen)			[left=of examen]				{\PK{idExamen}};
		\node[attribute]	(examenFechaCreacion)		[right=of examenDificultad]		{fechaCreacion};
		\node[attribute]	(examenTexto)				[right=of examen]			{texto};
		
		%%Atributos programador
		\node[attributePK]	(programadorIdProgramador)	[left=of programador]		{\PK{idProgramador}};
		\node[attribute]	(programadorNombre)			[below left=of programador]		{nombre};
		\node[attribute]	(programadorApellido)		[below=of programador]			{apellido};
		
		%%Atributos evalua
		\node[attribute]	(conoceNota)	[below=of conoce]							{nota};
		
		
		\path
		%%Atributos examen
		(examen)		edge 	(examenIdExamen)
		(examen)		edge 	(examenCantEjercicios)
		(examen)		edge 	(examenDificultad)
		(examen)		edge	(examenFechaCreacion)
		(examen)		edge	(examenTexto)
		
		%% Atributos lenguaje
		(lenguaje)		edge 	(lenguajeIdLenguaje)
		(lenguaje)	 	edge 	(lenguajeNombre)
		
		%%Atributos programador
		(programador)	edge 	(programadorIdProgramador)
		(programador)	edge 	(programadorNombre)
		(programador)	edge 	(programadorApellido)
		
		%% Atributos conoce
		(conoce)		edge	(conoceNota)
		
		%%Relacion conoce
		(conoce)		edge				node[above, near end]	{N}	(programador)
		(conoce)	edge	node[right, near end]{1}	(examen)	
		;
		
		\path[rightOptional]
		(conoce)	edge	node[above, near end]{1}	(lenguaje)
		
		;
		
		\end{tikzpicture}
	}
\end{center}

\begin{multicols}{2}
	
	\textbf{MER}
	
	Programador(\PK{idProgramador}, nombre, apellido)
	
	Lenguaje	(\PK{idLenguaje}, nombre)
	
	Examen	(\PK{idExamen}, cantEjercicios, dificultad, fechaCreacion, texto)
	
	conoce(\FPK{idProgramador}, \FPK{idLenguaje}, \FK{examen}, nota)
	
	\columnbreak
	\red{\textbf{Notas}}
	\begin{itemize}
		\item Las relaciones ternarias pueden tener atributos siempre y cuando\textbf{ no } sean identificatorios
	\end{itemize}
	\null\vfill
\end{multicols}

\newpage
\subsubsection*{b)}
\begin{center}
	\textbf{DER}
	\vspace*{1cm}
	
	\scalebox{1}{
		\begin{tikzpicture}[.style={DER}]
		
		%%Entidades y relaciones
		\node[entity]		(programador)												{Programador};
		\node[relationship] (conoce)					[right=of programador]			{conoce};
		\node[entity]		(lenguaje)					[right=of conoce]				{Lenguaje};
		\node[relationship]	(evaluaA)					[above=of programador]			{evalua a};
		\node[entity]		(examen)					[above=of evaluaA]				{Examen};
		\node[relationship] (esSobre)					[above=of lenguaje]				{es sobre};
		
		%%Atributos examen
		\node[attribute]	(examenDificultad)			[above= of examen]				{dificultad};
		\node[attribute]	(examenCantEjercicios)		[left=of examenDificultad]		{cantEjercicios};
		\node[attributePK]	(examenIdExamen)			[left=of examen]				{\PK{idExamen}};
		\node[attribute]	(examenFechaCreacion)		[right=of examenDificultad]		{fechaCreacion};
		\node[attribute]	(examenTexto)				[below left=of examen]			{texto};
		
		%%Atributos lenguaje
		\node[attributePK]	(lenguajeIdLenguaje)		[right=of lenguaje]			{\PK{idLenguaje}};
		\node[attribute]	(lenguajeNombre)			[below right=of lenguaje]		{nombre};
		
		%%Atributos programador
		\node[attributePK]	(programadorIdProgramador)	[left=of programador]		{\PK{idProgramador}};
		\node[attribute]	(programadorNombre)			[below left=of programador]		{nombre};
		\node[attribute]	(programadorApellido)		[below=of programador]			{apellido};
		
		%% Atributos evaluaA
		\node[attribute]	(evaluaANota)				[right=of evaluaA]			{nota};
		
		\path
		%% Atributos examen
		(examen)		edge 	(examenIdExamen)
		(examen)		edge 	(examenCantEjercicios)
		(examen)		edge 	(examenDificultad)
		(examen)		edge	(examenFechaCreacion)
		(examen)		edge	(examenTexto)
		
		%%Aributos lenguaje
		(lenguaje)		edge 	(lenguajeIdLenguaje)
		(lenguaje)	 	edge 	(lenguajeNombre)
		
		%%Atributos programador
		(programador)	edge 	(programadorIdProgramador)
		(programador)	edge 	(programadorNombre)
		(programador)	edge 	(programadorApellido)
		
		%% Atributos evalua a
		(evaluaA)		edge	(evaluaANota)
		
		%%Relacion conoce
		(conoce)	edge	node[above, near end]	{N}	(programador)
		
		%%Relacion es sobre
		(esSobre)	edge[in=east, out=north]	node[above, pos=0.95]	{M}	(examen)
		;
		
		\path[rightOptional]
		%% Relacion conoce
		(conoce)	edge	node[above, near end]	{M}	(lenguaje)
		
		%% Relación evalua a
		(evaluaA)	edge	node[right, near end]	{N}	(programador)
		(evaluaA)	edge	node[right, near end]	{N}	(programador)
		(evaluaA)	edge	node[right, near end]	{M}	(examen)
		
		%% Rleacion es Sobre
		(esSobre)	edge	node[right, near end]	{1}	(lenguaje)
		;
		\end{tikzpicture}
	}
\end{center}

\begin{multicols}{2}
	
	\textbf{MER}
	
	Programador(\PK{idProgramador}, nombre, apellido)
	
	Lenguaje	(\PK{idLenguaje}, nombre)
	
	Examen	(\PK{idExamen}, cantEjercicios, dificultad, fechaCreacion, texto, \FK{idLenguaje})
	
	conoce(\FPK{idProgramador}, \FPK{idLenguaje})
	
	evaluaA(\FPK{idExamen}, \FK{idProgramador}, nota)
	
	\columnbreak
	\textbf{Restricciones adicionales:}
	\begin{itemize}
		\item Los examenes que evaluan a un programador deben ser sobre los lenguajes que conoce.
	\end{itemize}
\end{multicols}

