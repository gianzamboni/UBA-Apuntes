%Considere una base de datos que registra los resultados obtenidos por los estudiantes en diferentes exámenes de diferentes cursos realizados.
%\begin{enumerate}
%	\item Construya el DER y MR suponiendo que Examen es una entidad y utilizando relaciones ternarias.
%	
%	\item Construya una alternativa usando sólo una relación binaria entre Estudiante y Curso. Asegúrese de que sólo existe una relación
%	entre un estudiante en particular y un curso, pero puede representar los resultados que un estudiante obtiene en diferentes exámenes.
%\end{enumerate} 


\subsubsection*{a)}
\begin{center}
	\textbf{DER}
	\vspace*{1cm}
	
	\scalebox{1}{
		\begin{tikzpicture}[.style={DER}]
		%% Entidades; y relaciones
		\node[entity]		(examen)										{Examen};
		\node[relationship] (calificaA)			[right=of examen]			{califica a};
		\node[entity] 		(alumno)			[right=of calificaA]		{Alumno};
		\node[entity]		(curso)				[below=of calificaA]		{Curso};
		
		%% Atributos alumno
		\node[attributePK]	(alumnoIdAlumno)	[above=of alumno]			{\PK{idAlumno}};
		\node[attribute]	(alumnoNombre)		[above right=of alumno]		{nombre};
		\node[attribute]	(alumnoApellido)	[right=of alumno]			{apellido};
		
		%% Atributos curso
		\node[attributePK]	(cursoIdCurso)		[left=of curso]				{\PK{idCurso}};
		\node[attribute]	(cursoNombre)		[right=of curso]			{nombre};
		
		%% Atributos examen
		\node[attributePK]	(examenIdExamen)	[above=of examen]			{\PK{idExamen}};
		\node[attribute]	(examenFecha)		[above left=of examen]		{fecha};
		\node[attribute]	(examenNota)		[left=of examen]			{nota};
		
		\path	
		%%Atributos alumno
		(alumno)	edge	(alumnoIdAlumno)
		(alumno)	edge	(alumnoNombre)
		(alumno)	edge	(alumnoApellido)
		
		%%Atributos curso
		(curso)		edge	(cursoIdCurso)
		(curso)		edge	(cursoNombre)
		
		%% Atributos examen
		(examen)	edge	(examenIdExamen)
		(examen)	edge	(examenFecha)
		(examen)	edge	(examenNota)
		
		%% Relacion califica a
		(calificaA)	edge	node[above, near end] {N}	(examen)
		(calificaA)	edge	node[above, near end] {1}	(alumno)
		(calificaA) edge	node[right, near end] {1}	(curso)
		;		
		\end{tikzpicture}
		
	}
	
\end{center}

\vspace*{1cm}
\begin{multicols}{2}
	
	\textbf{MER}
	
	Examen	(\PK{idExamen}, fecha, nota)
	
	Curso	(\PK{idCurso}, nombre)
	
	Alumno	(\PK{idAlumno}, nombre, apellido)
	
	califica a(\FPK{idAlumno}, \FPK{idExamen}, \FK{idCurso})
	
	\columnbreak
	
	\paragraph{Nota:} Este modelo permite que una materia tome más de un parcial a un alumno en la misma fecha. ¿Como arreglaría esto?.
	
\end{multicols}
\newpage
\subsubsection*{b)}
\begin{center}
	\textbf{DER}
	\vspace*{1cm}
	
	\scalebox{1}{
		\begin{tikzpicture}[.style={DER}]	
		%% Entidades y realaciones
		\node[entity]		(curso)														{Curso};
		\node[relationship]	(evaluaA)			[right=of curso]						{califica a};
		\node[entity]		(alumno)			[right=of evaluaA]						{Alumno};
		
		%% Atributos alumno			
		\node[attributePK]	(alumnoIdAlumno)	[above=of alumno]						{\PK{idAlumno}};
		\node[attribute]	(alumnoNombre)		[above right=of alumno]					{nombre};
		\node[attribute]	(alumnoApellid)		[right=of alumno]						{apellido};
		
		%% Atributos curso
		\node[attributePK]	(cursoIdCurso)		[above=of curso]						{\PK{idCurso}};
		\node[attribute]	(cursoNombre)		[left=of curso]							{nombre};
		
		%% Atributos evalua a
		\node[attributePK] 	(evaluaFecha)		[below left=of evaluaA]					{\PK{fecha}};
		\node[attribute] 	(evaluaNota)		[below  right=of evaluaA]				{nota};
		
		\path	
		%% Atributos alumno
		(alumno)			edge	(alumnoIdAlumno)
		(alumno)			edge	(alumnoNombre)
		(alumno)			edge	(alumnoApellido)
		
		%%Atributos curso
		(curso)				edge	(cursoIdCurso)
		(curso)				edge	(cursoNombre)
		
		%%Atributos evaluaA
		(evaluaA)	edge	node[above, near end] {N} (alumno)
		(evaluaA) edge	node[above, near end] {M} (curso)
		
		%%Relacion evalua A
		(evaluaA) edge	(evaluaFecha)
		(evaluaA) edge	(evaluaNota)
		;		
		
		\end{tikzpicture}
		
	}
	
\end{center}

\vspace*{1cm}

\begin{multicols}{2}
	\textbf{MER}
	
	Examen	(\PK{idExamen}, fecha, nota)
	
	Curso	(\PK{idCurso}, nombre)
	
	califica a(\FPK{idAlumno}, \FPK{idCurso}, \PK{fecha}, nota)
	
	\columnbreak
	
	\paragraph{Nota:} Este modelo soluciona el problema del anterior. Ahora una materia solo puede tomar un exámen por fecha a cada alumno.
\end{multicols}
