\documentclass[10pt,a4paper]{article}

\usepackage{algpseudocode}
\usepackage{algorithmicx}
\usepackage{amsfonts}
\usepackage{amsmath}
\usepackage{amssymb}
\usepackage[spanish]{babel}
\usepackage{colortbl}
\usepackage{dsfont}
\usepackage{environ}
\usepackage[shortlabels]{enumitem}
\usepackage{fancyhdr}
\usepackage{fourier}
\usepackage{graphicx}
\usepackage[hidelinks]{hyperref}
\usepackage{ifthen}
\usepackage[utf8]{inputenc}
\usepackage{lscape}
\usepackage{multicol}
\usepackage{minted}
\usepackage{tabularx}
\usepackage{titlesec}
\usepackage{titling}
\usepackage{tikz}
\usepackage[normalem]{ulem}
\usepackage{xcolor}
\usepackage{varwidth}
\usepackage{verbatim}
\usepackage{wrapfig}

\usepackage{geometry}

\title{Bases de datos - Práctica 2}
\author{}

%%%% CONFIGURACIONES %%%%

%% La coma de los reales es un punto
\decimalpoint

%%% Tamaño de pagina
\geometry{
	includeheadfoot,
	margin=1cm
}

%\stul{0.1cm}{0.2ex}

%% HEADER Y FOOTER
\pagestyle{fancy}

\titleformat{\subsection}
{\normalfont\Large\bfseries}{}{1em}{}

\renewcommand\thesubsubsection{\alph{subsubsection}}

\fancyhf{}


\fancyhead[LO]{Sección \rightmark} % \thesection\ 
\fancyhead[RO]{\small{\thetitle}}
\fancyfoot[CO]{\thepage}
\renewcommand{\headrulewidth}{0.5pt}
\renewcommand{\footrulewidth}{0.5pt}
\setlength{\headsep}{1cm}
\setlength{\headheight}{13.07225pt}

\renewcommand{\baselinestretch}{1.2}  % line spacing

%% Links en indice 
\hypersetup{
	linktoc=all,     %set to all if you want both sections and subsections linked
	linkcolor=blue,  %choose some color if you want links to stand out
}

\setlength{\columnsep}{50pt}

\newcommand{\black}[1]{{\color{black}#1}}%\renewcommand{\appendixtocname}{Apéndices}

\newcommand{\PK}[1]{\underline{#1}}
\newcommand{\FK}[1]{\dashuline{#1}}
\newcommand{\FPK}[1]{\PK{\FK{#1}}}

\newcommand{\red}[1]{{\color{red}#1}}

\definecolor{TableHeader}{rgb}{0.813, 0.856, 0.931}


\newcommand{\warningsign}{\tikz[baseline=0.1cm] \node[shape=regular polygon, regular polygon sides=3, inner sep=1pt, draw, thick] {\textbf{!}};}


\tikzstyle{atentionBox} = [
draw=gray, very thick,
inner xsep=0.5cm, inner ysep=0.5cm, draw=red!90, color=red!80!black!150, fill=red!5!white!80
]

\NewEnviron{atencion}[1]{%
	\begin{center}
		\begin{tikzpicture}
		\node [atentionBox](box){%
		\scalebox{#1}{\warningsign}\hspace*{0.5cm} 
		\begin{minipage}{0.8\textwidth}
		\Large \BODY		
		\end{minipage}
		};
		\end{tikzpicture}
	\end{center}
}

\begin{document}
\usetikzlibrary{positioning}
\usetikzlibrary{shapes.geometric}
\usetikzlibrary{arrows}
\usetikzlibrary{arrows.meta}
\usetikzlibrary{fit}
\usetikzlibrary{calc}
\usetikzlibrary{matrix}

\tikzstyle{DER}=[
	node distance=1cm
]

\tikzstyle{DERM}=[
	row sep=1cm,
	column sep=1cm
]

\tikzstyle{agregacion} = [
rectangle,
draw=blue!70!black!80,
align=left,
color=blue!50!black!100,
text=black,
very thick, 
minimum width=6em, 
minimum height=3em
]

\tikzstyle{entity} = [
	rectangle,
	draw=blue!70!black!80,
	align=left,
	color=blue!50!black!100,
	text=black,
	fill=blue!15!white!100, 
	very thick, 
	minimum width=6em, 
	minimum height=3em
]

\tikzstyle{weak entity} = [
	entity,
	double, 
	double distance=2pt
]

\tikzstyle{relationship} = [
	diamond,
	align=center,
	draw=red!80,
	color=red!80!black!150,
	text=black,
	fill=red!15!white!100,
	very thick,
	minimum width=2em,	
	aspect=2
]

\tikzstyle{weak relationship} = [
	relationship, 
	double,
	double distance=2pt,
]

\tikzstyle{rightOptional} = [-{Circle[open, length=5].Square[width=0.5]}]

\tikzstyle{leftOptional} = [{Square[width=0.5].Circle[open,length=5]}-]

\tikzstyle{attribute} = [
	ellipse, 
	draw=black,
	very thick,
	minimum width=5em,
	minimum height=2em
]

\tikzstyle{attributePK} = [
	attribute,
	draw=orange!80!black!100,
	fill=orange!15!white!100,
	text=black
]

\tikzstyle{attributeM} = [
	attribute,
	double,
	double distance=2pt
]

\tikzstyle{inheritance} = [
	circle,
	draw=black,
	very thick,
]

\maketitle
\tableofcontents
\newpage
\section{AR y CRT}
\setlength{\columnsep}{0pt}
\subsection{Ejercicio 1.1}
\begin{multicols}{4}
		\paragraph{a)} $R\cup S$
	
	\hspace*{0.063cm}
	\begin{tabular}{| l | l |}
		\hline
		\rowcolor{TableHeader}\textbf{A} &  \textbf{B} \\
		\hline
		a & b\\
		\hline
		b & c\\
		\hline
		b & d\\
		\hline
		c & b\\
		\hline
		d & e\\
		\hline
		e & a\\
		\hline
	\end{tabular}
	
	\vfill\null
	
	\columnbreak
	\paragraph{b)} $R-S$

\hspace*{0.063cm}
\begin{tabular}{| l | l |}
	\hline
	\rowcolor{TableHeader}\textbf{A} &  \textbf{B} \\
	\hline
	a & b\\
	\hline
	c & b\\
	\hline
	d & e\\
	\hline
\end{tabular}

\vfill\null

\columnbreak
	\paragraph{c)} $R\times S$

\hspace*{0.063cm}
\begin{tabular}{| l | l | l | l |}
	\hline
	\rowcolor{TableHeader}\textbf{A} &  \textbf{B} & \textbf{S.B} &  \textbf{C}\\
	\hline
	a & b & b & c\\
	\hline
	a & b & e & a\\
	\hline
	a & b & b & d\\
	\hline
	b & c& b & c\\
	\hline
	b & c& e & a\\
	\hline
	b & c& b & d\\
	\hline
	c & b& b & c\\
	\hline
	c & b& e & a\\
	\hline
	c & b& b & d\\
	\hline
	d & e& b & c\\
	\hline
	d & e& e & a\\
	\hline
	d & e& b & d\\
	\hline
\end{tabular}

\columnbreak
\paragraph{d)} $R\bowtie S$

\hspace*{0.063cm}
\begin{tabular}{| l | l | l | l |}
	\hline
	\rowcolor{TableHeader}\textbf{A} &  \textbf{B} &   \textbf{C}\\
	\hline
	a & b & c \\
	\hline
	a & b & d \\
	\hline
	c & b & c \\
	\hline
	c & b & d\\
	\hline
	d & e & a\\
	\hline
\end{tabular}
\end{multicols}

\begin{multicols}{4}
	\paragraph{e)} $\pi_B(R)$
	
	\hspace*{0.063cm}
	\begin{tabular}{| l | }
		\hline
		\rowcolor{TableHeader} \textbf{B} \\
		\hline
		 b\\
		\hline
		 c\\
		\hline
		 e\\
		\hline
	\end{tabular}
	
	\vfill\null
	
	\columnbreak
	\paragraph{f)} $\sigma_{A=C}(R\times S)$
	
\hspace*{0.063cm}
\begin{tabular}{| l | l | l | l |}
	\hline
	\rowcolor{TableHeader}\textbf{A} &  \textbf{B} & \textbf{S.B} &  \textbf{C}\\
	\hline
	a & b & e & a\\
	\hline
	c & b& b & c\\
	\hline
\end{tabular}	
	\vfill\null
	
	\columnbreak
	\paragraph{g)} $S \div (T \bowtie S)$
	
	\hspace*{0.063cm}
	\red{$T \bowtie S$ es una relacion con dos tuplas de $S$, ¿que pasa aca?}
	\vfill\null
	\columnbreak
	\paragraph{h)} $R\bowtie_{R.B < S.C} S$

\hspace*{0.063cm}
\begin{tabular}{| l | l | l | l |}
	\hline
	\rowcolor{TableHeader}\textbf{A} &  \textbf{B} &   \textbf{C}\\
	\hline
	a & b & c \\
	\hline
	a & b & d \\
	\hline
	c & b & c \\
	\hline
	c & b & d\\
	\hline
\end{tabular}
\end{multicols}
\begin{multicols}{2}
\subsection{Ejercicio 1.2}
\begin{itemize}
	\item[] $\rho(R_1,~R)$
	\item[] $\rho(a \rightarrow a_1,~R_1)$
	\item[] $\rho(RES,~R - \pi_a(\sigma_{a > a_1}(R\times R_1)))$
\end{itemize}

	\subsection{Ejercicio 1.3}
	\begin{itemize}
		\item[] $\rho(R_1,~R)$
		\item[] $\rho(a \rightarrow a_1,~R_1)$
		\item[] $\rho(RES,~R - \pi_{a, b}(\sigma_{a = a_1 \land b > b_1}(R\times R_1)))$
	\end{itemize}
	
	\subsection{Ejercicio 1.4}
	\begin{itemize}
		\item[] $\rho(R_1,~R)$
		\item[] $\rho(a \rightarrow a_1,~R_1)$
		\item[] $\rho(RES4,~\pi_{a, b}(\sigma_{a = a_1 \land b \neq b_1}(R\times R_1)))$
	\end{itemize}
	
	\subsection{Ejercicio 1.5}
	\begin{itemize}
		\item[] $\rho(R_1,~R)$
		\item[] $\rho(a \rightarrow a_1,~R_1)$
		\item[] $\rho(R_2,~R)$
		\item[] $\rho(a \rightarrow a_2,~R_2)$
		\item[] $\rho(RES,~\pi_{a, b}(\sigma_{a = a_1 \land b \neq b_1 \land a = a_2 \land b \neq b_2 \land b_1 \neq b_2}(R\times R_1\times R_2)))$
	\end{itemize}	
\end{multicols}

\newpage
	\subsection{Ejercicio 1.6}
\paragraph{CRT}
\begin{itemize}
	\item[] $\rho(JOIN,~ \text{invoiceline} \bowtie \text{invoice} \bowtie \text{customer})$
	\item[] $\rho(JOIN_1,~JOIN)$
	\item[] Renombramos todos los attributos de $JOIN_1$ para que no haya incertidumbre agregandoles un 1 al final.
	\item[] $\rho(PROD, JOIN \times JOIN_1)$
	\item[] $\rho(RES,~\pi_{\text{first\_name}}(PROD - \sigma_{\text{quantity} < \text{quantity}_1}(PROD)))$
\end{itemize}

\paragraph{AR}
\begin{itemize}
	\item [] $\{~t~/ \exists~il,i,c~( il\in \text{invoiceline}~\land~i\in \text{invoice}~\land~c\in \text{customer}~\land~ \text{esMaximaCantidad}(il)~\land~\text{esDelCliente}(il,i,c)\\ ~\land~t.\text{nombre} = c.\text{firstName})\}$
	\item[] $\text{esMaximaCantidad}(il) = \forall~il_1~( il_1\in \text{invoiceline} \Rightarrow il.\text{quantity} > il_1.\text{quantity})$
	\item[] $\text{esDelCliente}(il,i,c) = il.\text{invoiceId} = i.\text{invoiceId}~\land~ i.\text{customerId} = c.\text{customerId}$
\end{itemize}
\section{AR, CRT y SQL}
\subsection{Ejercicio 2.1}
	\subsubsection{}
\paragraph{SQL}
\begin{minted}[tabsize=4]{sql}
SELECT 
	first_name,
	last_name
FROM customer
WHERE country = 'Brazil';
\end{minted}
\paragraph{AR}
\begin{itemize}
	\item[] $\pi_{\text{first\_name, last\_name}}(\sigma_{\text{country} = 'Brazil'}(\text{customer}))$
\end{itemize}

\paragraph{CRT} 
\begin{itemize}
	\item[] $\{ t / \exists~c~(c\in \text{customer} \land c.\text{country} = 'Brazil' \land t.\text{nombre} = c.\text{first\_name} \land t.\text{apellido} = c.\text{last\_name})\}$
\end{itemize}

\newpage
\subsubsection{}
\paragraph{SQL}
\begin{minted}[tabsize=4]{postgresql}
SELECT
	c.first_name as name,
	c.last_name as last_name,
	i.invoice_id as invoice_number,
	i.invoice_date as invoice_date
FROM
	customer c
	JOIN invoice i ON i.customer_id = c.customer_id
ORDER BY c.first_name;
\end{minted}
\paragraph{AR}
\begin{itemize}
	\item[] $\pi_{\text{first\_name, last\_name, invoice\_id, invoice\_date}}(\text{invoice} \bowtie \text{customer})$
\end{itemize}

\paragraph{CRT} 
\begin{itemize}
	\item[] $\{ t / \exists~c,i~(c\in \text{customer} \land i\in invoice \land c.\text{customer\_id} = i.\text{customer\_id} \\ \land t.\text{nombre} = c.\text{last\_name} \land t.\text{invoice\_id} = i.\text{invoice\_id} \land t.\text{fecha} = i.\text{invoice\_id})\}$
\end{itemize}

\subsubsection{}
\paragraph{SQL}
\begin{minted}[tabsize=4]{postgresql}
SELECT
	tr.name as track,
	ar.name as artist
FROM
	track tr
	JOIN album al on tr.album_id = al.album_id
	JOIN artist ar on ar.artist_id = al.artist_id;
\end{minted}
\paragraph{AR}
\begin{itemize}
	\item[] $\rho($name $\rightarrow$ track\_name, track)
	\item[] $\rho($name $\rightarrow$ artist\_name, artist)
	\item[] $\pi_{\text{track\_name, artist\_name}}($track $\bowtie$ album $\bowtie$ artist)
\end{itemize}

\paragraph{CRT} 
\begin{itemize}
	\item[] $\{ t / \exists~ar,~al,~tr~(tr\in \text{track} \land
		al\in\text{album} \land ar\in\text{artist} \\ 
		\land \text{esDelArtista}(tr,al,ar) \land
		t.\text{track\_name} = tr.\text{name} \land t.\text{artist\_name} = ar.\text{name}\}
		$
	\item[] esDelArtista($tr,al,ar$) = $tr.\text{album\_id} = al.\text{album\_id}\land al.\text{artist\_id} = ar.\text{artist\_id}$ 
\end{itemize}

\newpage
\subsubsection{}
%\paragraph{SQL}
%\begin{minted}{postgresql}
%SELECT 
%	pl.name
%FROM
%	playlist pl
%	JOIN playlist_track pt ON pl.playlist_id = pt.playlist_id
%	JOIN track tr ON tr.track_id = pt.track_id
%	JOIN media_type mt ON tr.media_type_id = mt.media_type_id
%GROUP BY pl.name
%HAVING COUNT(mt.name = 'MPEG audio file') > 1;
%
%\end{minted}
\paragraph{AR}
\begin{itemize}
	\item[] $\rho(\text{name}\rightarrow\text{pl\_name}, \text{playlist})$
	\item[] $\rho(\text{name}\rightarrow\text{tr\_name}, \text{track})$
	\item[] Consigo todos los tracks con formato MPEG que pertenecen a alguna lista:
	\item[] $\rho(MPEGS, \sigma_{\text{name} = '\textit{MPEG audio file}'}(\text{playlist} \bowtie \text{playlist\_track} \bowtie \text{media\_type})$
	\item[] $\rho(MPEGS_1, MPEGS)$
	\item[] Renombro todos los atributos de $MPEGS_1$ agregándoles un 1 al final.
	\item $\rho(RES, \pi_{\text{pl\_name}}(\sigma_{\text{name} \neq \text{name}_1 \land \text{pl\_name} = \text{pl\_name}_1}(MPEGS\times MPEGS_1)))$
\end{itemize}

\paragraph{CRT} 
\begin{itemize}
	\item[] $\{ t / \exists~pl,~tr_1,~tr_2,~pt_1,~pt_2, mt~(pl\in \text{playlist} \land tr_1,~tr_2\in\text{track} \land pt_1,~pt_2\in\text{playlist\_track} \land mt\in \text{media\_type} \\ 
	\land~\text{sonDistintosTracks}(tr_1, tr_2)~\land~ \text{pertenecenALaMismaLista}(tr_1,tr_2,pt_1,pt_2, pl) \\ ~\land~ 
	\text{tienenElMismoFormato}(tr_1, tr_2) ~\land~
	\text{esMPEG}(tr_1) ~\land~
	t.\text{name} = pl.\text{name}
	\}
	$
	\item[] sonDistintosTracks($tr_1,tr_2$) = $tr_1.\text{track\_id} \neq tr_2.\text{track\_id}\land al.\text{artist\_id} = ar.\text{artist\_id}$ 
	
	\item[] pertenecenALaMismaLista($tr_1,tr_2,pt_1,pt_2, pl$) = $pt_1.\text{playlist\_id} = pl.\text{playlist\_id}~\land~ pt_1.\text{track\_id} = tr_1.\text{track\_id}  \\~\land~ pt_2.\text{playlist\_id} = pl.\text{playlist\_id} ~\land~ pt_2.\text{track\_id} = tr_2.\text{track\_id}$
	
	\item[]
\end{itemize}

\subsubsection{}
\begin{minted}[tabsize=4]{postgresql}
SELECT pl.name
FROM
	playlist pl
	JOIN playlist_track pt ON pt.playlist_id = pl.playlist_id
	JOIN track tr ON tr.track_id = pt.track_id
	JOIN album al ON al.album_id = tr.album_id
	JOIN artist ar ON ar.artist_id = al.artist_id
GROUP BY pl.name
HAVING COUNT(ar.name='Iron Maiden') > 10;
\end{minted}

Es posible realziar esta consulta tanto en CRT como en AR. En CRT, implicaría realizar el join de todas las tabla y, luego, realizar la productoria de la relación resultante consigo misma 11 veces. En AR, de manera similar, deberiamos escribir implicitamente 11 comparaciones que indiquen si una lista, tiene por lo menos 11 tracks de iron maiden.

\subsubsection{}
\begin{minted}[tabsize=4]{postgresql}
SELECT
	pl.name as playlist_name,
	COUNT(DISTINCT album_id) as album_quantity
FROM
	playlist pl
	JOIN playlist_track pt ON pt.playlist_id = pl.playlist_id
	JOIN track tr ON tr.track_id = pt.track_id
GROUP BY pl.name;
\end{minted}

\subsubsection{}
\begin{minted}[tabsize=4]{postgresql}
SELECT em.first_name
FROM
	employee em
	JOIN invoice inv ON em.employee_id = inv.customer_id
	JOIN invoice_line il ON il.invoice_id = inv.invoice_id
WHERE date_part('year', age(em.birth_date)) > 25
GROUP BY em.employee_id
HAVING COUNT(*) > 10;
\end{minted}

\subsubsection{}
\begin{minted}[tabsize=4]{postgresql}
SELECT
	customer.first_name as name,
	invoice.invoice_id as invoice_number,
	invoice.invoice_date as invoice_date
FROM
	customer
	RIGHT OUTER JOIN invoice ON invoice.customer_id = customer.customer_id;
\end{minted}

\subsubsection{}
\begin{minted}[tabsize=4]{postgresql}
SELECT DISTINCT em.first_name
FROM
	employee em
	JOIN customer cu ON cu.support_rep_id = em.employee_id
WHERE cu.customer_id IN 
	(SELECT cu.customer_id
	 FROM
		invoice inv
		JOIN customer cu ON inv.customer_id = cu.customer_id
		GROUP BY cu.customer_id
		HAVING COUNT(*) > 10
	);
\end{minted}

\subsubsection{}
\begin{minted}[tabsize=4]{postgresql}
SELECT 
	em.first_name as employee_first_name,
	em.last_name as employee_last_name,
	bs.first_name as boss_first_name,
	bs.last_name as boss_last_name
 FROM
 	employee em
	JOIN employee bs ON em.reports_to = bs.employee_id;
\end{minted}

\newpage
\subsubsection{}
\begin{minted}[tabsize=4]{postgresql}
SELECT 
	em.first_name as employee_first_name,
	em.last_name as employee_last_name,
	bs.first_name as boss_first_name,
	bs.last_name as boss_last_name
FROM
	employee em
	LEFT OUTER JOIN employee bs ON em.reports_to = bs.employee_id;
\end{minted}

\subsubsection{}
\begin{minted}[tabsize=4]{postgresql}
SELECT 
	cu.customer_id,
	cu.first_name,
	cu.last_name,
	COUNT(*) / COUNT(DISTINCT inv.invoice_id) as quantity
FROM 
	customer cu
	JOIN invoice inv ON cu.customer_id = inv.customer_id
	JOIN invoice_line il ON il.invoice_id = inv.invoice_id
GROUP BY cu.customer_id;
\end{minted}

\subsubsection{}
\begin{minted}[tabsize=4]{postgresql}
SELECT
	em.employee_id,
	em.first_name as employee_first_name,
	em.last_name as employee_last_name,
	COUNT(mt.name = 'Rock') as rock_tracks_selled
FROM
	employee em
	JOIN customer cu ON em.employee_id = cu.support_rep_id
	JOIN invoice inv ON inv.customer_id = inv.customer_id
	JOIN invoice_line il ON il.invoice_id = inv.invoice_id
	JOIN track tr ON il.track_id = tr.track_id
	JOIN media_type mt ON tr.media_type_id = mt.media_type_id
GROUP BY em.employee_id;
\end{minted}

\subsection{Ejercicio 2.2}
\subsubsection{}
\paragraph{SQL}
\begin{minted}[tabsize=4]{postgresql}
SELECT DISTINCT s.bar
FROM
	sirve s
	JOIN gusta g ON s.cerveza = g.cerveza
WHERE g.persona = 'Juan K';
\end{minted}

\paragraph{CRT}
\begin{itemize}
\item [] $\pi_{\text{bar}}(\sigma_{\text{persona} = \text{'Juan K'}}(\text{GUSTA} \bowtie \text{SIRVE}))$
\end{itemize}
\paragraph{AR}
\begin{itemize}
\item[] $\{t~/~\exists g,s (s\in\text{SIRVE} \land g\in\text{GUSTA} \land g.\text{persona} = \text{'Juan K'}\land g.\text{cerveza} = s.\text{cerveza} \land t.\text{bar} = s.\text{bar})\}$
\end{itemize}
\subsubsection{}
\paragraph{SQL}
\begin{minted}[tabsize=4]{postgresql}
SELECT s.persona
FROM 
	sirve s
	JOIN gusta g ON g.cerveza = s.cerveza
	JOIN frecuenta f ON s.bar = f.bar AND g.persona = f.persona;
\end{minted}


\paragraph{CRT}
\begin{itemize}
\item [] $\rho(RES22B,~\pi_{\text{persona}}(\text{GUSTA} \bowtie \text{SIRVE} \bowtie \text{FRECUENTA}))$	
\end{itemize}

\paragraph{AR}
\begin{itemize}
\item[] $\{t~/~\exists g,s,f (s\in\text{SIRVE} \land g\in\text{GUSTA} \land f\in\text{FRECUENTA} \\ \land g.\text{cerveza} = s.\text{cerveza} \land s.\text{bar} = f.\text{bar} \land  g.\text{persona} = g.\text{persona})\}$
\end{itemize}
\subsubsection{}
\paragraph{SQL}
\begin{minted}[tabsize=4]{postgresql}
SELECT persona
FROM frecuenia
EXCEPT 
	(SELECT s.persona
	 FROM 
		sirve s
		JOIN gusta g ON g.cerveza = s.cerveza
		JOIN frecuenta f ON s.bar = f.bar AND g.persona = f.persona
	)
\end{minted}


\paragraph{CRT}
\begin{itemize}
\item[] $\pi_{\text{persona}}(FRECUENTA) - RES22B$
\end{itemize}

\paragraph{AR}
\begin{itemize}
	\item[] $\{ t / \exists~ g~ (g\in\text{GUSTA} \land \text{noFrecuentaNingunBarQueSirvaAlgoQueLeGuste}(g) \land t.\text{persona} = g.\text{persona})\} $
	
	\item[] $\text{noFrecuentaNingunBarQueSirvaAlgoQueLeGuste}(g) = \forall~ s~(\text{esUnBarQueLeGusta}(s,g) \Rightarrow noFrecuenta(g,s))$
	
	\item[] esUnBarQueSirveAlgoQueLeGusta$(s,g) = s\in SIRVE \land \exists~g1~(g1\in\text{GUSTA} \land g1.\text{persona} = g.\text{persona} \land s.\text{cerveza} = g1.\text{cerveza})$
	
	\item[] noFrecuenta$(s,g) = \nexists~f~(f\in\text{FRECUENTA} \land f.\text{bar} = s.\text{bar}\land f.\text{persona} = g.\text{persona})$
\end{itemize}

\subsubsection{}
\paragraph{SQL}
\begin{minted}[tabsize=4]{postgresql}
SELECT f.persona
FROM frecuenia f
GROUP BY f.persona
HAVING COUNT(*) = (
	SELECT COUNT(DISTINCT s.bar)
	 FROM sirve s
	)
\end{minted}


\paragraph{CRT}
\begin{itemize}
	\item[] $\rho(TodosLosBares, \pi_{\text{bar}}(\text{SIRVE}))$
	\item[] $\text{FRECUENTA} \div~TodosLosBares$
\end{itemize}

\paragraph{AR}
\begin{itemize}
	\item[] $\{ t / \exists~ f~ (f\in\text{FRECUENTA} \land \text{frecuentaTodosLosBares}(f) \land t.\text{persona} = f.\text{persona})\} $
	
	\item[] $\text{frecuentaTodosLosBares}(f) = \forall~ s~(s\in\text{SIRVE} \Rightarrow \\ \exists~f1 (f1\in\text{FRECUENTA} \land f1.persona = f.persona \land s.bar = f1.bar)))$
\end{itemize}

\subsection{Ejercicio 2.3}
\subsubsection{}
\begin{minted}[tabsize=4]{postgresql}
SELECT
	al.album_id,
	al.title
FROM
	album al
	JOIN track tr ON tr.album_id = al.album_id
	JOIN playlist_track pt ON pt.track_id = tr.track_id
GROUP BY al.album_id
HAVING COUNT(DISTINCT pt.playlist_id) = (
	SELECT COUNT(*) 
	FROM playlist
	)	
\end{minted}

\newpage
\subsubsection{}
\begin{minted}[tabsize=4]{postgresql}
SELECT
	al.album_id,
	al.title
FROM
	album al
	JOIN track tr ON tr.album_id = al.album_id
	JOIN playlist_track pt ON pt.track_id = tr.track_id
GROUP BY al.album_id
HAVING COUNT(DISTINCT pt.playlist_id) = 
	(SELECT COUNT(*) 
	FROM playlist
	)	
\end{minted}

\subsubsection{}
\begin{minted}[tabsize=4]{postgresql}
SELECT
	ar.name
FROM
	artist ar
	JOIN album al ON al.artist_id = ar.artist_id
	JOIN track tr ON al.album_id = tr.album_id
	JOIN playlist_track pt ON pt.track_id = tr.track_id
GROUP BY ar.name
HAVING COUNT(DISTINCT al.album_id) = (
	SELECT MIN(ql.pl_albums)
	FROM
		(SELECT
			ar.name, 
			COUNT(DISTINCT al.album_id) as pl_albums
		FROM 
			artist ar
			JOIN album al ON al.artist_id = ar.artist_id
			JOIN track tr ON tr.album_id = al.album_id
			JOIN playlist_track pt ON pt.track_id = tr.track_id
		GROUP BY ar.name) ql
	)
\end{minted}

\newpage
\subsection{Ejercicio 2.4}
\subsubsection{}
\paragraph{SQL}
\begin{minted}[tabsize=4]{postgresql}
SELECT pl.name
FROM
	playlist pl
	JOIN playlist_track pt ON pl.playlist_id = pt.playlist_id
	JOIN track tr ON tr.track_id = pt.track_id
	JOIN album al ON al.album_id = tr.album_id
	JOIN artist ar ON ar.artist_id = al.artist_id
GROUP BY pl.name
HAVING 
	COUNT(ar.name = 'Black Sabbath') = 0 
	AND COUNT(ar.name = 'Chico Buarque') = 0;
\end{minted}

\paragraph{AR}
\begin{itemize}
\item[] $\rho(\text{name} \to \text{playlist\_name}, \text{ playlist})$
\item[] $\rho(\text{name} \to \text{track\_name}, \text{ track})$
\item[] $\rho(\text{name} \to \text{artist\_name}, \text{ artist})$
\item[] $\rho(todaInfo,~\text{playlist} \bowtie \text{playlist\_track} \bowtie \text{track} \bowtie \text{artist})$
\item[] $\rho(listasSabathBuarque,~\sigma_{\text{artist\_name = 'Black Sabbath' } \lor \text{ artist\_name = 'Chico Buarque'}}(todaInfo))$
\item[] $\pi_{\text{playlist\_name}}(playlist) - \pi_{\text{playlist\_name}}(listaSabbathBuarque)$
\end{itemize}

\paragraph{CRT}
\begin{itemize}
\item[] $\{ t~/~\exists pl~(pl\in\text{playlist} \land t.\text{name} = pl.\text{name} \land \text{noTieneSabbathNiBuarque}(pl))\}$

\item[] noTieneSabbathNiBuarque$(pl) = \forall pt~(pt\in\text{playlitst\_track}\land pt.\text{playlist\_id} = pl.\text{playlist\_id} \Rightarrow \nexists~tr(tr\in\text{track} \land tr.\text{track\_id} = pt.\text{track\_id} \land \text{esDeSabatthOBuarque}(tr)))$

\item[] esDeSabbathOBuarque$(tr) = \exists ar,al~ (ar\in\text{artist} ~\land~ al\in\text{album} ~\land~ tr.\text{album\_id} = al.\text{album\_id} \\ \land~ al.\text{artist\_id} = ar.\text{artist\_id} ~\land (ar.\text{artist\_name} = \text{'Black Sabbath' } \lor ar.\text{artist\_name} = \text{'Chico Buarque'}))$
\end{itemize}

\subsubsection{}
\paragraph{SQL}
\begin{minted}[tabsize=4]{postgresql}
SELECT
	cu.customer_id,
	cu.first_name,
	cu.last_name
FROM
	customer cu
	JOIN invoice inv ON inv.customer_id = cu.customer_id
	JOIN invoice_line il ON il.invoice_id = inv.invoice_id
	JOIN track tr ON tr.track_id = il.track_id
	JOIN genre gr ON gr.genre_id = tr.genre_id
GROUP BY cu.customer_id
HAVING COUNT(DISTINCT gr.genre_id) = 1
\end{minted}

\paragraph{AR}
\begin{itemize}
\item[] $\rho(\text{name} \to \text{track\_name}, \text{ track})$
\item[] $\rho(\text{name} \to \text{genre\_name}, \text{ genre})$
\item[] $\rho(todaInfo, \text{customer} \bowtie \text{invoice} \bowtie \text{invoice\_line} \bowtie \text{track} \bowtie \text{genre})$
\item[] $\rho(todaInfo_1,~todaInfo)$
\item[] Además renombro todos los atributos de $todaInfo_1$ agregandoles un $1$ al final.

\item[] $\rho(clientesMasDeUnGenero,~\pi_{\text{first\_name, last\_name}}(\sigma_{genre\_id_1 \neq genre\_id}(todaInfo \bowtie todaInfo_1)))$

\item[] $\pi_{\text{first\_name, last\_name}}(customer) - clientesMasDeUnGenero$
\end{itemize}

\paragraph{CRT}
\begin{itemize}
\item[] $\{ t~/\exists~cu,inv,il,tr,gr (\text{ pertenecenALasTablasQueTieneQuePertenecer}(cu,inv,il,tr,gr)\\ \land \text{ formanUnLindoJoin}(cu,inv,il,tr,gr) \land \\ \land \text{ noExisteOtroTrackDeEseGeneroEnNingunOtroInvoiceDeEseCliente}(cu,inv,il,tr,gr)\}$

\item[] pertenecenALasTablasQueTieneQuePertenecer$(cu,inv,il,tr,gr) = cu\in\text{customer} \land inv\in\text{invoice} \land il\in\text{invoice\_line}\land tr\in\text{track} \land gr\in\text{genre}  \\ \land t.\text{customer\_id} = cu.$

\item[] formanUnLindoJoin$(cu,inv,il,tr,gr) = cu.\text{customer\_id} = inv.\text{customer\_id} \land inv.\text{invoice\_id} = il.\text{invoice\_id} \\ \land
il.\text{track\_id}  = tr.\text{track\_id} \land tr.\text{genre\_id} = gr.\text{genre\_id}$

\item[] noExisteOtroTrackDeEseGeneroEnNingunOtroInvoiceDeEseCliente$(cu,inv,il,tr,gr) = \\ \nexists cu_1,inv_1,il_1,tr_1,gr_1 ( \text{pertenecenALasTablasQueTieneQuePertenecer}(cu_1,inv_1,il_1,tr_1,gr_1) \\ \land \text{formanUnLindoJoin}(cu_1,inv_1,il_1,tr_1,gr_1) \land cu.\text{customer\_id} = cu_1.\text{cutomer\_id} \land gr_1.\text{genre\_id} = \text{genre\_id}$
\end{itemize}
\subsection{Ejercicio 2.5}
\subsubsection{}
\begin{minted}[tabsize=4]{postgresql}
SELECT ac.nombreActor
FROM
	actor ac
	JOIN participa_en pe ON pe.idActor = ac.actorId
	JOIN serie s ON s.idSerie = pe.idSerie
WHERE
	ac.edad > 30
	AND se.nombreSerie = 'Friends';
\end{minted}

\newpage
\subsubsection{}
\begin{minted}[tabsize=4]{postgresql}
SELECT ca.nombreCanal
FROM
	canal ca
	JOIN transmite tr ON tr.idCanal = ca.idCanal
	JOIN serie s ON s.idSerie = tr.idSerie
	JPON genero g ON g.idGenero = s.idGenero
GROUP BY ca.idCanal
HAVING COUNT(g.idGenero = 'comedia') = (
	SELECT COUNT(*)
	FROM
		serie s
		JOIN genero g ON g.idGenero = s.idGenero
	GROUP BY g.nombreGenero
	HAVING g.nombreGenero = 'comedia'
	)

\end{minted}

\subsubsection{}
\begin{minted}[tabsize=4]{postgresql}
SELECT	ac.nombreActor
FROM
	actor ac
	JOIN particia_en pe ON pe.idActor = ac.idActor
	JOIN serie s ON s.idSerie = p.idSerie
WHERE 
	ac.edad > 30
	AND s.nombreSerie = 'Friends'
	AND ac.idActor IN 
		(SELECT ac.idActor
		FROM 
			actor ac
			JOIN participa_en pe ON pe.idActor = ac.idActor
			JOIN serie s ON s.idSerie = pe.idSerie
		WHERE s.añoInicio > 2000
		)
\end{minted}

\newpage
\subsubsection{}
\begin{minted}[tabsize=4]{postgresql}
SELECT ac.nombreActor
FROM
	actor ac
	JOIN particia_en pe ON pe.idActor = ac.idActor
	JOIN serie s ON s.idSerie = p.idSerie
WHERE 
	ac.edad > 30
	AND s.nombreSerie = 'Friends'
	AND ac.idActor NOT IN 
		(SELECT ac.idActor
		 FROM 
			actor ac
			JOIN participa_en pe ON pe.idActor = ac.idActor
			JOIN serie s ON s.idSerie = pe.idSerie
		WHERE s.añoInicio > 2000
		)
\end{minted}

\subsubsection{}
\begin{minted}[tabsize=4]{postgresql}
SELECT idSerie
FROM serie
WHERE añoInicio = (
	SELECT MAX(añoInicio)	
	FROM serie
	)
\end{minted}

\subsubsection{}
\begin{minted}[tabsize=4]{postgresql}
SELECT acDos.idActor
FROM (
	SELECT
		ac.idActor as idActor,
		COUNT(DISTINCT pe.idSerie) as cantSeries
	FROM
		actor ac
		JOIN participa_en pe ON pe.idActor = ac.idActor
	GROUP BY ac.actorId) acDos
WHERE acDos.cantSeries >= 2
\end{minted}

\subsubsection{}
\begin{minted}[tabsize=4]{postgresql}
SELECT seDos.nombre
FROM (
	SELECT
		s.nombre as nombre,
		COUNT(DISTINCT pe.idSerie) as cantSeries
	FROM
		serie s
	GROUP BY s.nombre) acDos
WHERE sDos.cantSeries >= 2
\end{minted}
\subsubsection{}
\begin{minted}[tabsize=4]{postgresql}
SELECT tr.idCanal
FROM trasmite tr
WHERE tr.idSerie IN (
	SELECT s.idSerie 
	FROM serie s
	WHERE s.nombreSerie IN (
		SELECT seDos.nombre
		FROM (
			SELECT
				s.nombreSerie as nombre,
				COUNT(DISTINCT pe.idSerie) as cantSeries
			FROM
				serie s
				GROUP BY s.nombre) acDos
			WHERE sDos.cantSeries >= 2
	)
)
\end{minted}

\subsubsection{}
\begin{minted}[tabsize=4]{postgresql}
SELECT s.idSerie
FROM
	serie s
	JOIN participa_en pe ON pe.idSerie = s.idSerie
	JOIN actor ac ON ac.idActor = pe.idActor
GROUP BY s.idSerie
HAVING AVG(ac.edad) > (
	SELECT MAX(sq.promedio)
	FROM (
		SELECT AVG(ac.edad) as promedio
		FROM
			serie s
			JOIN participa_en pe ON pe.idSerie = s.idSerie
			JOIN actor ac ON ac.idActor = pe.idActor
		GROUP BY s.idSerie
		) sq
	)
\end{minted}

\newpage
\subsubsection{}
\begin{minted}[tabsize=4]{postgresql}
SELECT g.idGenero
FROM
	genero g
	JOIN serie s ON s.idGenero = g.idGenero
	JOIN participa_en pe ON pe.idSerie = s.idSerie
WHERE pe.idActor IN (
	SELECT ac.idActor
	FROM actor ac
	WHERE ac.edad = (
		SELECT MIN(ac.edad)
		FROM actor ac
	)
)
\end{minted}

\subsection{\red{Ejercicio 2.6 - Falta}}
\subsection{\red{Ejercicio 2.7 - Falta}}
\subsection{\red{Ejercicio 2.8 - Falta}}
\subsection{\red{Ejercicio 2.9 - Falta}}


\end{document}